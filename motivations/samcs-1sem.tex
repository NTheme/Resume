% -------- MOTIVATIONS --------
\section{\textbf{Наставник Будузих Физтехов. Мотивационное письмо}}
\smallskip

\quad Добрый день!
\smallskip

\quad Я Никитин Артем Анатольевич, закончил в этом году 11 класс Лицея <<Вторая школа>> поступил на ФПМИ ИВТ СППМ с индивидуальным планом на
ПМИ, имея БВИ как призер заключительного этапа ВсОШ по физике за 10 класс. Помимо этого, был участником финала ВсОШ по математике и стал
призером по экономике в том же году. На информатику пройти, к сожалению, не удалось, не хватило 13 из 800 баллов на региональном этапе. Почти
все лето после 10 класса было потрачено на отбор на межнар по физике, я вошел в топ-20 по России по рейтингу (могу предоставить официальную
таблицу). Также я прошел на летнюю смену между 10 и 11 классом в ЛКШ в параллель A', но не смог поехать как раз из-за отбора на межнар. Затем,
начиная с 11 класса, сделал упор на математику и информатику, так как решил, что дальше быть в физике не хочется. Начиная с 9 класса, я
занимаюсь на кружках Команды Москвы по математике и состою в самой сильной (1ой) группе, задачи оттуда и продолжал решать до самого всероса
этого года. Последнее достижение в физике было в конце ноября - серебро за 11 класс на одной из международных олимпиад: International
Experimental Physics Olympiad (IEPhO). В середине учебного года ездил на Январскую школу в Сириус по математике и программированию, на отборе
на нее набрал максимальное количество баллов по обеим конкурсным группам (как математика, так и информатика). Участвовал в различных командных
олимпиадах, таких, как МКОШП или Открытый Финал Московских Тренировок, получая диплом 2 степени. Занимался в кружках от Тинькофф в параллели А'
в 11 классе, ходил в ЦПМ как кандидат в сборную Москвы. Также прошел на Майскую смену по проектной математике и информатике в Сириус в этом 
году, но не поехал по состоянию здоровья.

\quad В своей школе я вел кружок по олимпиадной информатике для 6-7 классов (основы алгоритмов) и для 8-9 (уровень В-В’ Тинькофф), так что имею
опыт в работе с учениками. Знания алгоритмов и их математических оснований вполне достаточно для проведения занятий в рамках ФПМИ-кружков. В
ближайшем семестре ребята на моих занятиях будут узнавать нетривиальные способы решения и сдавать подобранные мной задачи в контестах и турах,
как тематических, так и олимпиадных. Физ-игры – это своего рода турнир по физике для олимпиадников из регионов в рамках их подготовки к
настоящим соревнованиям, в том числе и ко всеросу. В этом семестре я буду готовить задачи, проверять решения, указывать ребятам на более легкие
способы, разбирать и рассказывать новые методы. Будет возможность двигать сильно заинтересованных ребят дальше, подбирая им задачи и темы более
высокого уровня (в том числе всероса и выше, у меня есть доступ к закрытым материалом, да и составлять свои задачи не составит труда). 

\begin{flushright}
    С уважением, \\
    Артем Никитин
\end{flushright}
