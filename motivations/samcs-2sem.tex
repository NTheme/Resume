% -------- MOTIVATIONS --------
\section{\textbf{Наставник Будузих Физтехов. Мотивационное письмо}}
\smallskip

\quad Добрый день!
\smallskip

\quad Я Никитин Артем Анатольевич, группа Б05-232, выпускник ГБОУ <<Лицей <<Вторая школа>> имени В.Ф. Овчинникова, сейчас студент 1 курса МФТИ
ФПМИ ИВТ СППМ с индивидуальным планом на ПМИ. Когда я учился в школе, я занимался олимпиадной физикой, стал призером заключительного этапа и
кандидатом в национальную сборную, математикой на кружках от ЦПМ (подготовка кандидатов в сборную Москвы), также был на заключительном этапе,
и информатикой на кружках от Тинькоффа. Начал преподавать уже в 11ом классе: вёл кружок по олимпиадной информатике для 6 и 7 классов в школе.
Помогать и объяснять материал мне нравилось всегда: хочется не просто узнавать самому что-то новое, но и передавать полученный опыт,
самостоятельно разработанные идеи и способы решения другим ребятам, чтобы облегчить их подготовку, и самое главное – развить в детях интерес к
предмету. А в прошедшем семестре активно проводил занятия со школьниками из регионов России по олимпиадной информатике в рамках программы
ФПМИ-кружков. В течение 5 прошедших месяцев я готовил учеников из Республики Башкортостан, Волгоградской области к муниципальному и
региональному этапам ВсОШ, а также к отборочным перечневых, успел рассказать различные алгоритмы на графах, бинарные деревья и деревья поиска,
необходимую базовую математику. Начиная с нового года провел двухнедельный интенсив в формате зачета для подготовки к региональному этапу ВсОШ,
по результатам которого лучшим ученикам отправил призы. Школьникам такой формат понравился, на следующий семестр планирую проводить подготовку
прошедших на заключительный этап в том же формате с интенсивами, а для не прошедших – занятия по подготовке к перечневым, таким, как <<Открытая
олимпиада школьников по программированию>>, олимпиада <<Технокубок>>, олимпиада <<Высшая проба>> и другие. В перспективе запустить направление
промышленного программирования, основная цель которого будет повышение интереса младших школьников к информатике в целом. Формат занятий: дети
дописывают почти готовые и понятные куски кода и видят результат, что увеличивает их желание заниматься и дальше. Мне хочется передать
полученный опыт ребятам и показать, что программирование --– это не только многочасовое продумывание алгоритмов и отладка, но еще и очень много
интересного с точки зрения самого процесса, возможности создать работающее приложение из ничего. А в олимпиадном направлении – повысить уровень
школьников и помочь им достичь успехов в олимпиадах, указать на ошибки и предостеречь против них, продвинуть заинтересованных ребят вперед,
став для них наставником, в том числе сделав ФПМИ одним из самых значимых на их пути.

\quad Также в прошедшем семестре помогал с составлением материалов для кружка по олимпиадной физике в Республике Башкортостан. Род деятельности
не ограничивается исключительно преподаванием: в этом учебном году я проводил муниципальный этап ВсОШ по математике в г. Москва, а в новом
семестре, уже в ближайшие несколько недель, буду помогать на региональных этапах по математике и физике в этом же регионе (подтверждающие
документы смогу предоставить в начале февраля). Так же поеду в составе жюри на заключительный этап ВсОШ по физике в апреле, утверждающий состав
приказ будет в конце февраля. Искренне надеюсь увидеть своих учеников на финале!

\begin{flushright}
    С уважением, \\
    Артем Никитин
\end{flushright}
