% -------- MOTIVATIONS --------
\section{\textbf{Наставник Будузих Физтехов. Мотивационное письмо}}
\smallskip

\quad Добрый день!
\smallskip

\quad Я Никитин Артем Анатольевич, группа Б05-232, в прошлом ученик ГБОУ <<Лицей <<Вторая школа>> имени В.Ф. Овчинникова, дипломант
заключительных этапов ВсОШ по физике и экономике, сейчас студент 2 курса МФТИ ФПМИ ПМИ. Свою преподавательскую деятельность начал еще в 11
классе: тогда я в школе вел кружок по олимпиадной информатике для 6 и 7 классов, где самостоятельно разрабатывал план обучения.

\quad Учить чему-то и объяснять материал мне нравилось всегда: хочется не просто узнавать самому что-то новое, но и передавать полученный опыт,
самостоятельно разработанные идеи и способы решения следующим поколениям школьников, чтобы облегчить их подготовку, и самое главное – развить в
детях интерес к предмету. А в олимпиадном направлении стремлюсь повысить уровень школьников и помочь им достичь успехов в олимпиадах, указать
на ошибки и предостеречь против них, продвинуть заинтересованных ребят вперед, став для них наставником, в том числе сделав ФПМИ одним из самых
значимых на их пути.

\quad На протяжении 1 курса обучения вел дистанционный кружок в рамках кружков Игоря Батманова по олимпиадной информатике для детей из
различных регионов России (Республика Башкортостан, Волгоградская область, Московская область…), который планирую продолжать и в этом году.
Среди моих выпускников за прошлый год есть победители МОШ, призеры Всесибирской олимпиады. Занятия включали в себя лекции, контесты,
тренировочные туры и периодические интенсивы по подготовке школьников к региональным и заключительным этапам олимпиад. На днях стартует отбор,
в этом году хочу набрать две параллели детей: старшую и младшую, а если будут совсем сильные ребята, которым окажется скучно, то и для них
организовать свою, чтобы продвигать вперед и их.

\quad Во втором семестре побывал на сборах Школы Спортивного Программирования ЯНАО в Ноябрьске, и планирую поехать туда же еще раз в
предстоящем, говорил по этому поводу с организатором смены и Борисом Крохиным: хотим сделать более организованную и развернутую популяризацию
ФПМИ для детей. В прошлый раз я был вынужден ограничиться лишь устными рассказами, ответами на вопросы и показом презентации, но, что радует,
ученики были заинтересованы в том, чтобы стремиться и попасть в будущем к нам.

\quad Также, уже летом, ездил на Летнюю Многопредметную Школу, проводимую ГБОУ <<Лицей <<Вторая школа>>, с которой сотрудничает ФПМИ. Она
проводилась в том числе для детей, для которых вопрос о поступлении наиболее важен (поступающих в 11 класс), перед которым стоит выбор
университета, и я посчитал своей обязанностью показать, что на ФПМИ могут попасть и найти для себя огромное количество полезного не только
математики, но еще и информатики, физики, экономисты. Ребята в том числе расспрашивали о том, а можно ли будет параллельно преподавать, и я с
радостью рассказал им про наличие поддержки и кружков.

\quad В августе проводилась Выездная школа ФПМИ в г. Уфе по математике и физике, длившаяся полторы недели, на которой я, к сожалению, не смог
присутствовать очно, но составлял листочки с задачами и оказывал помощь в организации дистанционно. В ее рамках была проведена
популяризационная программа, о которой договаривались так же с Борисом Крохиным.

\quad Помимо школ участвовал в проверке и апелляции регионального этапа ВсОШ по математике в г. Москве и Московской области, ММО. Участвовал в
проведении олимпиад МосОблМат на базе Физтех-Лицея. Сейчас участвую в проведении отбора на кружки Tinkoff Generation. Надеюсь снова увидеть
детей, у которых преподавал, как на кружках, так и на олимпиадах!

\begin{flushright}
    С уважением, \\
    Артем Никитин
\end{flushright}
