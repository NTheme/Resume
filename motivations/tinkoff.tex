% -------- MOTIVATIONS --------
\section{\textbf{Тинькофф.Образование. Мотивационное письмо}}
\smallskip

\quad Добрый день!
\smallskip

\quad Я Никитин Артем Анатольевич, выпускник ГБОУ <<Лицей <<Вторая школа>> имени В.Ф. Овчинникова, в настоящем закончил 1 курс МФТИ ФПМИ ПМИ.

\quad Учась в школе, я был дипломантом заключительного этапа ВсОШ по физике, экономике, участником по математике. На информатику пройти, к
сожалению, не удалось, не хватило 7 из 800 баллов на региональном этапе в 10 классе, а в 11 из программирования больше занимался промышленным,
но участвовал в занятиях параллели A' Тинькофф. Прошел на летнюю смену между 10 и 11 классом в ЛКШ в A', но не смог поехать из-за отбора на
международную олимпиаду по физике (серебро за 11 класс на International Experimental Physics Olympiad). В середине учебного года ездил на
Январскую школу в Сириус по математике и программированию, на отборе набрал максимальное количество баллов по обеим конкурсным группам.
Участвовал в различных командных олимпиадах, таких, как ВКОШП или Открытый Финал Московских Тренировок, получая диплом 2 степени. Начиная с 9
класса занимался в кружках Команды Москвы по математике в самой сильной (1ой) группе.

\quad Начал преподавать уже в 10 классе: вел кружок по олимпиадной информатике для 6 и 7 классов в школе, в 11 – для 8-9, кто имел шансы
становиться призером региона и выходить дальше.

\quad В этом году проводил занятия со школьниками из регионов России по олимпиадной информатике в рамках программы ФПМИ-кружков. В течение 5
прошедших месяцев я готовил учеников из Республики Башкортостан, Волгоградской области, Тюменской области, Краснодарского края к региональному
и заключительному этапам ВсОШ, а также к финалам перечневых. Промежуточной аттестацией на кружке за семестр был недельный интенсив и итоговый 
зачет, состоящий из трех компонент: работы в семестре, итогового контеста и теоретических задач.

\quad В марте был преподавателем старшей группы (уровень прохода на заключительный этап ВсОШ) на сборах в ЯНАО от регионального центра 
спортивного программирования.

\quad Род взаимодействия с детьми не ограничивался исключительно преподаванием: в этом году я участвовал в проведении и проверке муниципального 
и регионального этапа ВсОШ по математике в г. Москва и Московской области. Помогать и объяснять материал мне нравилось всегда: хочется не 
просто узнавать самому что-то новое, но и передавать полученный опыт, самостоятельно разработанные идеи и способы решения другим ребятам, чтобы 
облегчить их подготовку, структурировать и сделать ее более доступной.

\begin{flushright}
    С уважением, \\
    Артем Никитин
\end{flushright}


% -------- ACHIEVMENTS --------
\section{\textbf{Лидерские навыки}}
\resumeSubHeadingListStart

\resumePOR
{}{Организация олимпиадных сборов в г. Уфа при поддержке ФПМИ МФТИ}{}

\resumePOR
{}{Организация кружков на летней многопрофильной школе от Лицея <<Вторая школа>>}{}

\resumePOR
{}{Организация школьных спектаклей, помощь в организации выпускного в 11 классе}{}

\resumePOR
{}{Организация командной работы при написании школьных проектов (видеоплеер, игры, нейросети)}{}

\resumeSubHeadingListEnd


\section{\textbf{Достижения}}
\resumeSubHeadingListStart

\resumePOR
{}{Призёр заключительного этапа ВсОШ по физике}{}

\resumePOR
{}{Кандидат в национальную сборную на международную олимпиаду IPhO}{}

\resumePOR
{}{Серебро International Experimental Physics Olympiad}{}

\resumePOR
{}{Призёр заключительного этапа ВсОШ по экономике}{}

\resumePOR
{}{Двукратный дипломант II степени на ВКОШП}{}

\resumePOR
{}{Дипломант II степени Открытого финала московских тренировок}{}

\resumePOR
{}{Участник заключительного этапа ВсОШ по математике}{}

\resumePOR
{}{Выпускник параллели A’ Тинькофф Поколения}{}

\resumePOR
{}{Преподаватель кружка по олимпиадному программированию Тинькофф Поколения}{}

\resumePOR
{}{Преподаватель кружка по олимпиадному программированию ФПМИ МФТИ}{}

\resumePOR
{}{Участник заключительного этапа олимпиады Технокубок}{}

\resumePOR
{}{Трёхкратный призёр регионального этапа ВсОШ по информатике}{}

\resumePOR
{}{Участник Открытой олимпиады школьников по программированию}{}

\resumePOR
{}{Абрамовский стипендиат МФТИ (за успехи в учёбе)}{}

\resumeSubHeadingListEnd


\section{\textbf{Сложный проект}}
\resumeSubHeadingListStart

\resumePOR
{}{Сетевая игра “Battleship” с поддержкой видеоядра, многопоточки, звука и коммуникации}{}

\resumePOR
{}{Боты для Telegram, файловый менеджер, видеоплеер, сайты}{}

\resumePOR
{}{Утилиты для работы с регулярными языками и структурами данных}{}

\resumeSubHeadingListEnd


\section{\textbf{Владение языками}}
\resumeSubHeadingListStart

\resumePOR
{}{C/C++ (5): крупные проекты, основной язык, опыт с SFML, FFMpeg, DirectShow, реализации STL}{}

\resumePOR
{}{Python (4): Telegram-боты, файловый менеджер; опыт с qt, keras, pandas, tensorflow, tkinter, aiogram}{}

\resumePOR
{}{HTML \& JavaScript (3): реализация сайта для анализа игр в <<Мафию>>}{}

\resumePOR
{}{Базовые умения в Assembler x86 и ARM}{}

\resumePOR
{}{Умение пользоваться Bash}{}

\resumePOR
{}{Репозитории: \href{https://github.com/NTheme}{\logobox{faGithub}{GitHub}}}{}

\resumeSubHeadingListEnd
