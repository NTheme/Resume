% -------- MOTIVATIONS --------
\section{\textbf{Киберпротект. Мотивационное письмо}}
\smallskip

\quad Добрый день!
\smallskip

\quad Я студент МФТИ ФПМИ Артем Никитин. Окончил с отличием 2 курса и сейчас обучаюсь на 3ем, имея по итогам средний балл 9 из 10. Еще с
детства у меня был интерес и увлечение IT-сферой, компьютерами, программированием. Мне было любопытно, как работает все внутри, а использование
готовых решений вызывало во мне много идей по доработке и совершенствованию, расширению функционала. Операционные системы, способы их защиты и
хранения данных, работа с сетевым оборудованием и оптимизация процессов, на тот момент еще только в личных целях как, например, локальный
сервер с данными или распределенная навигационная система, было для меня важной задачей с точки зрения минимизации затраченного времени, сил и
упрощения взаимодействия с продуктом одновременно с сохранением функционала. С получением новых знаний в процессе обучения в университете
возможности стали расширяться, а благодаря изучению информационной безопасности и осознанию важности надежного хранения, я стал лучше понимать
принципы работы отказоустойчивых распределенных систем, что побудило меня всерьез задуматься о развитии и карьере в этом направлении.

\quad Я хотел бы поступить на кафедру Теоретической и прикладной информатики, так как уверен, что здесь могу глубже изучить интересующие меня
темы, освоить передовые технологии и реализовать свои наработки, продвинув данную область вперед. Я вижу, как важны решения по автоматизации и
прогнозированию неисправностей для предотвращения потенциальных проблемы, и хотел бы в том числе попробовать разработать и внедрить
использование глубокого обучения для более эффективных реализаций.

\quad Мой путь в программировании начался с изучения базовых алгоритмов во время участия в олимпиадах в школьные годы. Постепенно от олимпиад
я перешел к более прикладным задачам, и с момента начала изучения UNIX-систем параллельно с Windows начал осознавать, что наибольший интерес
для меня представляет углубление в мониторинг и анализ данных для защиты инфраструктур. Я вижу огромную важность данной задачи как для простых
пользователей с точки зрения повседневного использования, так и для крупных компаний, работа которых неотъемлемо связана с обеспечением
cтабильности и возможности быстрых откатов если вдруг что-то пойдет не так.

\quad Тем не менее, мои интересы не ограничиваются исключительно областью разработки. В школьные годы я занимался физикой, стал медалистом
международной олимпиады. Отчасти это мне помогло лучше понимать принципы работы компьютеров и систем на физическом уровне. Сейчас во время
учебы я прохожу много математических дисциплин, что помогает мне развивать навыки генерации идей и улучшать фундаментальное понимание. Но
особый отклик я получаю от направлений, связанных с программированием, распределенными задачами, операционными системами, методами оптимизации
и машинное обучением, курсы по которым я прохожу в данный момент. Конечно, обучение проходит не без трудностей: предметы, связанные с
абстрактными математическими конструкциями, многообразиями и функциональным анализом на топологиях для меня являются несколько проблемными, но
я активно над этим работаю, продолжаю развиваться дальше и надеюсь применять полученные знания на практике.

\quad На этом и основан мой выбор кафедры Теоретической и прикладной информатики в сотрудничестве с компанией Киберпротект. Для меня имеют
особый отклик проекты, связанные с поиском аномалий в сети и система S.M.A.R.T..

\quad К сожалению, в списке доступных нам на 3 курсе кафедр я не обнаружил чего-то, что бы являлось близким к предлагаемому кафедрой ТиПИ.
Выбор данного направления я рассматривал как основной еще летом перед 1 курсом обучения. Безусловно, в списке предложенных имеется много
связанных с программированием и IT, такие, как ИСП РАН или Кафедра вычислительной информатики, но в них нет того сочетания системного
программирования, низкоуровневых и распределенных систем, облачной архитектуры и возможности применения на прикладных задачах в сфере
разработки и безопасности. Я надеюсь стать частью разработок и команды, трудящейся над современными решениями в области защиты данных.

\quad В силу этого среди разработок Киберпротекта я считаю перспективными технологии восстановления данных после сбоя и гибкие инструменты
управления резервными копиями. Эти решения мне кажутся значимыми, потому что позволяют повысить надежность работы предприятий и минимизировать
потери данных, что особенно необходимо в современных условиях киберугроз. Я уже на своем опыте успел прочувствовать важность, когда на днях у
меня <<посыпался>> жесткий диск, но я успешно справился с восстановлением большей части потерянной информации. Так же обретенные знания уже 
были применены при использовании олимпиадной тестирующей системы для предотвращения взломов и гарантии отказоустойчивости работы и сохранности
данных и решений участников тренировочных сборов и олимпиад.

\quad В нашем мире данные являются ключевым ресурсом, и работа над резервным копированием и защитой информации - это не просто разработка
программных решений, а вклад в безопасность и надёжность информационных систем. Проекты кафедры подчеркивают, что я смогу вырасти как
специалист, решающий значимые прикладные задачи в этой области.

\quad Говоря о себе, среди моих сильных сторон - способность усваивать и систематизировать информацию, умение решать задачи и генерировать
новые идеи, методы, подходы к решению проблем, умение и опыт работы в команде. Я уже реализовывал как учебные, так и прикладные проекты
совместно с несколькими людьми, что помогло мне обрести опыт коммуникации, распределения задач и помощи коллегам в затруднениях. Также считаю
важным аспектом уже приобретенные за время обучения знания в перечисленных областях. Я работал со многими языками программирования, такими как
C/C++, assembler, python, go, java, html и другими, что позволяет мне иметь возможность реализации frontend и backend проектов разной
сложности. Фундаментальная база в виде математики и и основ программирования помогает мне использовать нетривиальные методы оптимизации и
различные алгоритмы. А пройденные курсы баз данных и параллельных вычислений являются необходимыми для осознания и погружения в методы
проектирования систем. Мне стоит еще поработать над навыками проектного управления и получить больше практики в реализации систем, чтобы
эффективнее вносить вклад в крупные проекты.

\quad Мой опыт и академические достижения на ФПМИ МФТИ дают мне хорошую базу для успешного освоения сложных технологий, которые применяются в
проектах кафедры. Буду рад возможности продолжить обучение на кафедре ТиПИ и работать над проектами, которые делают информационные системы
более безопасными и устойчивыми к сбоям. Спасибо за уделенное время.

\begin{flushright}
    С уважением, \\
    Артем Никитин
\end{flushright}
