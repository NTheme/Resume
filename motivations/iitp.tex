% -------- MOTIVATIONS --------
\section{\textbf{ИППИ РАН. Мотивационное письмо}}
\smallskip

\quad Добрый день!
\smallskip

\quad Я студент ФПМИ МФТИ Артем Никитин. Окончил с отличием 2 курса и сейчас обучаюсь на третьем, имея по итогам средний балл 9 из 10. Еще с
детства у меня был интерес и увлечение IT-сферой, компьютерами, программированием. Мне было любопытно, как работает все внутри, а использование
готовых решений вызывало во мне много идей по доработке и совершенствованию, расширению функционала. Но больше всего меня восхищало и удивляло
то, каким образом машины <<могут>> вести себя, видеть, слышать, распознавать мир способами, очень похожими на те, как это делаем мы. Тогда еще
я не знал, что внутри находятся алгоритмы и много лет упорной работы разработчиков, физиков и биологов, так что просмотр фильма Терминатор на
определенное время вселил в меня страх. Но знания, возможности расширялись в процессе обучения, а с ними и пришла большая ответственность,
заключающаяся в понимании того, сколько всего уже было изучено и разработано и сколько еще предстоит сделать на пути доработки и исследований.

\quad В средней школе родители подарили мне первый фотоаппарат. Поначалу мне нравилось фотографировать все без разбора, и, конечно же, порой
это не удавалось, ведь автоматические системы фокусировки, подбора экспозиции, выдержки, диафрагмы, светочувствительности и прочего далеко не
идеальны. А иногда получались кадры, вообще отличавшиеся от того, что я видел своими глазами на самом деле. И мне стало интересно понять, как
и почему это происходит, есть ли способ максимально приблизить имеющимися у меня тогда средствами реальное изображение к тому, что я вижу потом
на экране монитора. Пройдя курс теории фотографии, я стал лучше понимать устройство фотоаппарата и зрительного восприятия. Конечно же, мне в
этом еще помогло мое активное увлечение физикой (я стал медалистом международной олимпиады), ведь физическое понимание процессов является
неотъемлемой частью данного вопроса. В институте у меня появились знания для того, чтобы попробовать описать самостоятельно то, что <<видит>>
камера объектива, к тому же, я являюсь фотографом-волонтером студсовета и вопрос обработки фотографий, подбора оптимальных параметров и
осознания причин той или иной получившейся цветопередачи является повседневным. Потому я решил для себя, что мне хочется продолжать изучать эту
область, больше углубиться в нее, исследовать и пробовать применять различные алгоритмы и машинное обучение для более точного описания
цветового пространства и репродукции света. Я читал исследования в области технологий современной цветовой вычислительной фотографии и изучал
задачу о сравнении классических алгоритмов и нейронных сетей в вопросе цветовых преобразований, которая давалась на весеннем отборе, хотя и не
подавал заявку. Для меня задача описания цветовых пространств разных устройств и, в особенности, человека, являлась ключевой, хотя я и не мог
до определенного момента ее сформулировать формально. Я был удивлен, узнав, что на данный момент нейросети не могут добиться того значения
ошибки цветопередачи, которого достигают стандартные алгоритмы, и очень хотел бы продолжать изучение, разработки и вести исследования наряду с
тем, как это было осуществлено в предложенной весной статье, связанной с коррекцией света, результаты которой меня как раз и поставили в
замешательство.

\quad Тем не менее, мои увлечения не ограничиваются фотографией. Для меня также являются интересными задачи, связанные с распознаванием поз
человека, отчасти и в том числе потому, что эта область исследований связана с обработкой и анализом изображений. Я занимался 3D
моделированием во время учебы в школе, и основной проблемой была анимация персонажа посредствами компьютера. Практически невозможно было
сделать все движения естественными, а примерно похожие результаты требовали много времени. Потому мне хотелось углубиться в эту область и
изучить, а как можно максимально точно передать движения объектов в пространстве, сохранив это в виде цифровой записи. В том числе поражало,
что компьютерная графика в фильмах уже добилась больших успехов, хотя и до сих пор можно найти в ней много косяков. Но ведь в этом и состоит
задача: использовать свои силы и идеи, чтобы дорабатывать, изобретать новые подходы к решению вопроса. В том числе это может быть полезным и
в таких важных сферах, как медицина: на основе движений человека научиться определять некоторые заболевания или разрабатывать методики
восстановления после операций.

\quad С третьего класса я учился игре на гитаре, и продолжаю практиковаться по сей день, принимая участие в студенческих гитарниках или просто
что-то разучивая для себя. Это дело не могло обойтись без песен под мои исполнения: где-то мы собирались и выступали группой, где-то все было
сольным. И, разумеется, записывалось. И передо мной часто вставала задача отделения голоса от текста для более точного анализа результата или
эстетической корректировки отдельных частей записи для дальнейшего выкладывания. Или просто когда хотелось для практики перед выступлением из
цельной песни получить ее <<минусовку>>. Особенно трудным с этой точки зрения являлись композиции, в которых было несколько голосов. Потому
анализ и обработка звука для меня всегда так же являлись далеко не безразличной задачей. Как отделять голоса от нот или друг от друга? Как
наиболее точно очистить от шума или выделить определенную тональность? Со временем эти вопросы дополнились и переросли в связанное с тем, что
сейчас умеют генерировать нейросети. Я слушал песни, полностью написанные машиной, и переделанные оригиналы под другого исполнителя. Было очень
необычно осознавать, что в наушниках играет русская народная <<Калинка-малинка>>, но голосом метал-рок группы Sabaton, которая, разумеется, ее
никогда не пела. Я бы очень хотел лучше разобраться, как работают данные алгоритмы и, в особенности в том, как осознавать, где правда, а
где --- ложь, так называемый дипфейк. Я с большим желанием займусь исследованиями в области того, как машинам ловить тех же самых машин на
подобном обмане, ведь данные вопрос в том числе очень важен и с точки зрения информационной безопасности: сейчас в интернете очень много
сгенерированного (от видео на YouTube до речей известных личностей), что наивно принимают за чистую монету. В том числе и с точки зрения
видеоряда, анализом которых я бы был тоже рад заняться и реализовать свои идеи и наработки на практике.

\quad К сожалению, в списке доступных нам на 3 курсе кафедр я не обнаружил чего-то, что бы являлось близким к предлагаемому кафедрой ИППИ РАН.
Выбор данного направления я рассматривал как основной еще летом перед 1 курсом обучения. Безусловно, в списке предложенных имеется много
связанных с программированием и IT, но в них нет того сочетания современных методов, возможности концентрации на актуальных проблемах и задачах
и возможности исследования интересующих меня областей, как это есть у вас, наряду с написанием и публикацией научных работ для представления
результатов миру и из дальнейших продвижений.

\quad Говоря о себе, среди моих сильных сторон - способность усваивать и систематизировать информацию, умение решать задачи и генерировать
новые идеи, методы, подходы к решению проблем, умение и опыт работы в команде. Я уже реализовывал как учебные, так и прикладные проекты
совместно с несколькими людьми, что помогло мне обрести опыт коммуникации, распределения задач и помощи коллегам в затруднениях. Также считаю
важным аспектом уже приобретенные за время обучения знания в перечисленных областях. Я работал со многими языками программирования, такими как
C/C++, assembler, python, go, java, html и другими, что позволяет мне иметь возможность реализации frontend и backend проектов разной
сложности. Фундаментальная база в виде математики и и основ программирования помогает мне использовать нетривиальные методы оптимизации и
различные алгоритмы. А пройденные курсы нейронных сетей, баз данных и параллельных вычислений являются необходимыми для осознания и погружения
в методы, применяемые в областях цветопередачи, распознавания и анализа речи. Мой опыт и академические достижения на ФПМИ МФТИ подарили мне
умение генерации новых идей и продуктивного осознания новых областей. Буду рад возможности продолжить обучение на кафедре ИППИ РАН и работать
над проектами, которые помогут продвинуть данные области вперед на много шагов. Хотел бы сказать, что через 5 лет я стану <<гением,
миллиардером, плейбоем, филантропом>>, CEO огромной многомиллиардной компании, но время покажет. Сейчас я бы очень хотел расширить свои знания
и заняться первыми научными исследованиями, находящимися на стыке нескольких наук, что для меня всегда казалось очень перспективным,
увлекательным и интересным. И в скором будущем представить миру свои разработки в данных областях, оставив ранее неразрешимые задачи в прошлом,
или хотя бы стать частью той силы прогресса, которая движет знания, открытия и прогресс вперед.

\quad Спасибо за уделенное время.

\begin{flushright}
    С уважением, \\
    Артем Никитин
\end{flushright}
