% -------- MOTIVATIONS --------
\section{\textbf{Наставник Будузих Физтехов. Мотивационное письмо}}
\smallskip

\quad Добрый день!
\smallskip

\quad Я Никитин Артем Анатольевич, группа Б05-222, выпускник ГБОУ <<Лицея <<Вторая школа>>, дипломант
заключительных этапов ВсОШ по физике и экономике, сейчас студент 4 курса МФТИ ФПМИ ПМИ. Свою преподавательскую деятельность начал еще в
11 классе: тогда я в школе вел кружок по олимпиадной информатике для 6 и 7 классов, где самостоятельно разрабатывал план обучения.

\quad Учить чему-то и объяснять материал мне нравилось всегда: хочется не просто узнавать самому что-то новое, но и передавать полученный
опыт, самостоятельно разработанные идеи и способы решения следующим поколениям школьников, чтобы облегчить их подготовку, и самое главное –--
развить в детях интерес к предмету. А в олимпиадном направлении стремлюсь повысить уровень школьников и помочь им достичь успехов в олимпиадах,
указать на ошибки и предостеречь против них, продвинуть заинтересованных ребят вперед, став для них наставником, в том числе сделав ФПМИ одним
из самых значимых на их пути.

\quad Начиная с 1 семестра я веду дистанционный кружок ФПМИ в рамках кружков по олимпиадной информатике для детей из различных регионов России
(Республика Башкортостан, Волгоградская область и другие). Среди моих выпускников за прошлый год есть победители МОШ,
призеры Всесибирской олимпиады. Занятия включают в себя лекции, контесты, тренировочные туры и периодические интенсивы по подготовке школьников
к региональным и заключительным этапам олимпиад. Сейчас кружок стал более индивидуальным в силу количества людей, что дает мне возможность
разработать свой подход к каждому из моих учеников и довести их до поставленных ими себе целей.

\quad За прошедший семестр я дважды преподавал на региональных сборах в г. Тюмень по олимпиадному программированию - в январе и марте (даже во
время сессии). За сборы дети с увлечением решали интересные им задачи, с некоторыми сидели даже по несколько дней, но уходили
счастливыми от результата. В конце сборов мы, совместно с еще несколькими студентами ФПМИ, провели презентацию для всех участников и оставили
детям подарки на будущее.

\quad В течение семестра время я продолжал поддерживать работоспособность тестирущей системы для преподавателей и учеников
по программированию, с которой более подробно можно ознакомиться \href{https://algocode.madprogrammer.net/main/1/}{здесь}
(ссылка обновилась в силу миграции). Провел несколько масштабных обновлений, повысивших ее безопасность и стабильность.

\quad В конце января я преподавал и разрабатывал план подготовки для школьников на сборах перед региональным этапом по физике в Республике
Башкортостан. В рамках сборов мы помогали детям подготовиться к экспериментальному туру, рассказывали им нестандартные идеи и проводили с ними
тренировочные эксперименты. После чего они успешно выступили на самой олимпиаде (разглашать результаты не могу в данный момент, но вижу их как
член жюри этого региона).

\quad На протяжении всего семестра я вел и буду продолжать вести кружок по олимпиадному программированию Tinkoff Generation. Мы рассказываем
детям новые для них алгоритмы и методологии, помогаем научиться придумывать нестандартные идеи решения задач, просвещаем про открывающиеся
перед ними а будущем возможности, в том числе и про обучение в институте, и я с радостью отвечаю на все их вопросы про обучение и жизнь на
Физтехе, в частности про неоспоримые преимущества ФПМИ перед остальными. Ребятам, заканчивающим 11 класс в этом году, это очень актуально как
тем, для которых предстоит в ближайшем будущем один из важнейших выборов в их жизни ‐ в какой институт и на какое направление поступить. Буду
очень рад увидеть своих учеников в следующем году уже студентами моей Физтех-школы!

\quad Также, уже летом, ездил на Летнюю Многопредметную Школу, проводимую ГБОУ <<Лицей <<Вторая школа>>, с которой сотрудничает ФПМИ. Она
проводилась в том числе для детей, для которых вопрос о поступлении наиболее важен (поступающих в 11 класс), перед которым стоит выбор
университета, и я посчитал своей обязанностью показать, что на ФПМИ могут попасть и найти для себя огромное количество полезного не только
математики, но еще и информатики, физики, экономисты. Ребята в том числе расспрашивали о том, а можно ли будет параллельно преподавать, и я с
радостью рассказал им про наличие поддержки и кружков.

\quad В течение семестра участвовал в различных мероприятиях для школьников от ФПМИ как фотограф: буткемп, открытая олимпиада школьников по 
программированию, студенческая олимпиада.

\quad Помимо школ и кружков в прошлом семестре участвовал в организации, проверке и принятии апелляций регионального этапа ВсОШ по физике в Республике
Башкортостан - официально состоял в РПМК как член жюри. Так же помогал на проверке регионального этапа по математике ВсОШ в
г. Москва и Московской области, участвовал в проверке ММО в г. Москва. В апреле я сопровождал детей из Республики Башкортостан на
заключительный этап ВсОШ по физике. Во время самой олимпиады в г. Саранск я морально поддерживал участников, готовил к апелляции, которая в физике
является одной из важнейших частей. Надеюсь снова увидеть своих детей детей, у которых преподавал, как на кружках, так и на олимпиадах!

\quad В новом семестре планирую продолжать работать с детьми с кружка ФПМИ, кружок Tinkoff Образования. Поеду на сборы в Тюмень, которые пройдут в
октябре, где я буду обучать детей из этого и соседних регионов (ХМАО и ЯНАО) олимпиадному программированию. Среди
обучающихся будут так же и участники заключительного этапа ВсОШ по информатике. Во время сборов снова расскажем детям об МФТИ и в
частности ФПМИ, что повысит их интерес в участии в олимпиадном движении и
поступлении в нашу Физтех-школу как для тех, кто уже знает о нас, так и для тех, кто слышит впервые. Буду курировать проект по информационной
безопасности в рамках программы Сириус.Лето и ВсОШ по информатике. Надеюсь довести своих учеников до победы на заключительном этапе!
Приму участие в проверке муниципального этапа ВсОШ по математике в г. Москва. Также по возможности поеду в г. Ноябрьск,
где я уже преподавал прошлом году, но сейчас на городские сборы, которые помогут детям подготовиться к муниципальному этапу ВсОШ, сохранить
мотивацию и использовать ее, как и новообретенные знания, в полной мере в будущем году!

\begin{flushright}
    С уважением, \\
    Артем Никитин
\end{flushright}
