% -------- MOTIVATIONS --------
\section{\textbf{Наставник Будузих Физтехов. Мотивационное письмо}}
\smallskip

\quad Добрый день!
\smallskip

\quad Я Никитин Артем Анатольевич, группа Б05-222, в прошлом ученик ГБОУ <<Лицея <<Вторая школа>> имени В.Ф. Овчинникова, дипломант
заключительных этапов ВсОШ по физике и экономике, сейчас студент 3 курса МФТИ ФПМИ ПМИ. Свою преподавательскую деятельность начал еще в
11 классе: тогда я в школе вел кружок по олимпиадной информатике для 6 и 7 классов, где самостоятельно разрабатывал план обучения.

\quad Учить чему-то и объяснять материал мне нравилось всегда: хочется не просто узнавать самому что-то новое, но и передавать полученный
опыт, самостоятельно разработанные идеи и способы решения следующим поколениям школьников, чтобы облегчить их подготовку, и самое главное –--
развить в детях интерес к предмету. А в олимпиадном направлении стремлюсь повысить уровень школьников и помочь им достичь успехов в олимпиадах,
указать на ошибки и предостеречь против них, продвинуть заинтересованных ребят вперед, став для них наставником, в том числе сделав ФПМИ одним
из самых значимых на их пути.

\quad Начиная с 1 семестра я веду дистанционный кружок ФПМИ в рамках кружков по олимпиадной информатике для детей из различных регионов России
(Республика Башкортостан, Волгоградская область, Московская область и другие). Среди моих выпускников за прошлый год есть победители МОШ,
призеры Всесибирской олимпиады. Занятия включают в себя лекции, контесты, тренировочные туры и периодические интенсивы по подготовке школьников
к региональным и заключительным этапам олимпиад. Сейчас кружок стал более индивидуальным в силу количества людей, что дает мне возможность
разработать свой подход к каждому из моих учеников.

\quad В прошедшем семестре в январе участвовал в проведении сборов по олимпиадному программированию, организованных <<Сборником Олпрогера>>. На
них мы готовили детей к написанию олимпиады, которая для них уже случилась в ближайшем будущем: региональный этап ВсОШ по информатике. Ребята
заинтересовано в свободное время расспрашивали про то, почему я выбрал именно то место, где учусь сейчас. Детям также было любопытно, а можно
ли будет параллельно преподавать, и я с радостью рассказал им про наличие поддержки и кружков на ФПМИ.

\quad За прошедший семестр я дважды был на региональных сборах в г. Тюмень по олимпиадному программированию - в октябре и в январе (даже во
время сессии). За это время я успел разобраться и настроить полностью работоспособную тестирущую систему и удобный интерфейс для преподавателей
по программированию, с которым, как и с мероприятиями, в которых я принимал участие, более подробно можно ознакомиться здесь:
algofrog.ntheme.tech. За сборы дети с увлечением решали интересные им задачи, с некоторыми сидели даже по несколько дней, но уходили
счастливыми от результата.  В конце сборов мы провели презентацию ФПМИ для всех участников и оставили детям подарки на будущее.

\quad Также я побывал на сборах Школы Спортивного Программирования ЯНАО в Ноябрьске. В рамках прошедших сборов дети узнали несколько новых
алгоритмов, писали тренировочные туры, успешно справились с коллоквиумом и проверочными тестами, который были полностью подготовлены мной.
Была проведена подробная презентация нашей Физтех-Школы и, что радует, ученики были заинтересованы в том, чтобы стремиться и попасть в будущем
к нам. Многие остались в последний день после всех занятий чтобы поподробнее расспросить про жизнь и возможности, из-за чего я героически
чуть не опоздал на самолет.

\quad В конце января я преподавал и разрабатывал план подготовки для школьников на сборах перед региональным этапом по физике в Республике
Башкортостан. В рамках сборов мы помогали детям подготовиться к экспериментальному туру, рассказывали им нестандартные идеи и проводили с ними
тренировочные эксперименты. После чего они успешно выступили на самой олимпиаде (разглашать результаты не могу в данный момент, но вижу их как
член жюри этого региона).

\quad На протяжении всего семестра я вел и буду продолжать вести кружок по олимпиадному программированию Tinkoff Generation. Мы рассказываем
детям новые для них алгоритмы и методологии, помогаем научиться придумывать нестандартные идеи решения задач, просвещаем про открывающиеся
перед ними а будущем возможности, в том числе и про обучение в институте, и я с радостью отвечаю на все их вопросы про обучение и жизнь на
Физтехе, в частности про неоспоримые преимущества ФПМИ перед остальными. Ребятам, заканчивающим 11 класс в этом году, это очень актуально как
тем, для которых предстоит в ближайшем будущем один из важнейших выборов в их жизни ‐ в какой институт и на какое направление поступить. Буду
очень рад увидеть своих учеников в следующем году уже студентами моей Физтех-школы!

\quad Помимо школ и кружков в прошлом семестре участвовал в организации, проверке и апелляции регионального этапа ВсОШ по физике в Республике
Башкортостан - официально состоял в РПМК как член жюри. Так же сейчас помогаю на проверке и апелляции регионального этапа по математике в
г. Москве и Московской области. Помимо этого организовал и провел муниципальный этап ВсОШ по информатике в г. Ноябрьск - это было завершением
сборов. Надеюсь снова увидеть детей, у которых преподавал, как на кружках, так и на олимпиадах!

\quad В новом семестре планирую продолжать кружок по информатике ФПМИ, кружок Tinkoff. Поеду на сборы в Тюмень, которые пройдут в марте и будут
первыми выездными за все время, где я буду обучать детей из этого и соседних регионов (ХМАО и ЯНАО) олимпиадному программированию. Среди
обучающихся будут так же и участники заключительного этапа ВсОШ по информатике, для которых мы разрабатываем в рамках школы отдельную
программу. Во время сборов снова расскажем детям об МФТИ и в частности ФПМИ, что повысит интерес детей в участии в олимпиадном движении и
поступлении в нашу Физтех-школу как для тех, кто уже знает о нас, так и для тех, кто слышит впервые. Буду помогать с проверкой регионального
этапа по математике в двух регионах, завершу с принятием апелляций регионального этапа по физике в РБ, приму участие в проверке ММО и поеду на
заключительный этап ВсОШ по информатике амбассадором (в случае успешной организации со слов Бориса). Также по возможности поеду в г. Ноябрьск,
где я уже преподавал прошлом семестре, но сейчас на городские сборы, которые помогут не прошедшим на заключительный этап ВсОШ сохранить
мотивацию и использовать ее, как и новообретенные знания, в полной мере в будущем году!

\begin{flushright}
    С уважением, \\
    Артем Никитин
\end{flushright}
