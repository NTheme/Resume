% -------- MOTIVATIONS --------
\section{\textbf{Наставник Будузих Физтехов. Мотивационное письмо}}
\smallskip

\quad Добрый день!
\smallskip

\quad Я Никитин Артем Анатольевич, группа Б05-220, в прошлом ученик ГБОУ <<Лицей <<Вторая школа>> имени В.Ф. Овчинникова, дипломант
заключительных этапов ВсОШ по физике и экономике, сейчас студент 3 курса МФТИ ФПМИ ПМИ. Свою преподавательскую деятельность начал еще в 11
классе: тогда я в школе вел кружок по олимпиадной информатике для 6 и 7 классов, где самостоятельно разрабатывал план обучения.

\quad Учить чему-то и объяснять материал мне нравилось всегда: хочется не просто узнавать самому что-то новое, но и передавать полученный опыт,
самостоятельно разработанные идеи и способы решения следующим поколениям школьников, чтобы облегчить их подготовку, и самое главное – развить в
детях интерес к предмету. А в олимпиадном направлении стремлюсь повысить уровень школьников и помочь им достичь успехов в олимпиадах, указать
на ошибки и предостеречь против них, продвинуть заинтересованных ребят вперед, став для них наставником, в том числе сделав ФПМИ одним из самых
значимых на их пути.

\quad Начиная с 1 семестра я веду дистанционный кружок в рамках кружков Игоря Батманова по олимпиадной информатике для детей из различных
регионов России (Республика Башкортостан, Волгоградская область, Московская область и другие). Среди моих выпускников за прошлый год есть
победители МОШ, призеры Всесибирской олимпиады. Занятия включают в себя лекции, контесты, тренировочные туры и периодические интенсивы по
подготовке школьников к региональным и заключительным этапам олимпиад.

\quad В прошедшем семестре участвовал в проведении сборов по олимпиадному программированию, организованных “Сборником Олпрогера”. На них мы
готовили детей к написанию олимпиады, которая для них уже случилась в ближайшем будущем: региональный этап ВсОШ по информатике. Ребята
заинтересовано во свободное время расспрашивали про то, почему я выбрал именно то место, где учусь сейчас. Детям также было любопытно, а можно
ли будет параллельно преподавать, и я с радостью рассказал им про наличие поддержки и кружков.

\quad Также на протяжении всего семестра я вел и буду продолжать вести кружок по олимпиадному программированию Tinkoff Generation. Мы обучаем
детей новым алгоритмам, помогаем научиться придумывать нестандартные идеи решения задач, рассказываем про открывающиеся перед ними а будущем
возможности, в том числе и про обучение в институте, и я с радостью отвечаю на все их вопросы про обучение и жизнь на Физтехе, в частности про
неоспоримые преимущества ФПМИ перед остальными. Ребятам, заканчивающим 11 класс в этом году, это очень актуально как тем, для которых предстоит
в ближайшем будущем один из важнейших выборов в их жизни ‐ в какой институт и на какое направление поступить. Буду очень рад увидеть своих
учеников в следующем году уже студентами моей Физтех-школы!

\quad Помимо школ и кружков в прошлом семестре участвовал в проверке и апелляции регионального этапа ВсОШ по математике в г. Москве и области.
Надеюсь снова увидеть детей, у которых преподавал, как на кружках, так и на олимпиадах! Также проводил олимпиаду Физтех по математике и физике
для школьников 9-11 классов в г. Казань.

\quad В новом семестре планирую продолжать кружок по информатике ФПМИ, кружок Tinkoff. Поеду на сборы в Тюмень, которые пройдут уже в начале
октября, где я буду обучать детей из ФМШ олимпиадному программированию. Среди обучающихся будут так же и участники заключительного этапа ВсОШ
по информатике, для которых мы разрабатываем в рамках школы отдельную программу. В рамках самих сборов расскажем детям об МФТИ и в частности
ФПМИ, что повысит интерес детей в участии в олимпиадном движении и поступлении в нашу Физтех-школу. Также по возможности поеду в Воронеж, где
я уже преподавал прошлой осенью.

\begin{flushright}
    С уважением, \\
    Артем Никитин
\end{flushright}
